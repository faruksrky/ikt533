% Options for packages loaded elsewhere
\PassOptionsToPackage{unicode}{hyperref}
\PassOptionsToPackage{hyphens}{url}
%
\documentclass[
]{article}
\usepackage{lmodern}
\usepackage{amssymb,amsmath}
\usepackage{ifxetex,ifluatex}
\ifnum 0\ifxetex 1\fi\ifluatex 1\fi=0 % if pdftex
  \usepackage[T1]{fontenc}
  \usepackage[utf8]{inputenc}
  \usepackage{textcomp} % provide euro and other symbols
\else % if luatex or xetex
  \usepackage{unicode-math}
  \defaultfontfeatures{Scale=MatchLowercase}
  \defaultfontfeatures[\rmfamily]{Ligatures=TeX,Scale=1}
\fi
% Use upquote if available, for straight quotes in verbatim environments
\IfFileExists{upquote.sty}{\usepackage{upquote}}{}
\IfFileExists{microtype.sty}{% use microtype if available
  \usepackage[]{microtype}
  \UseMicrotypeSet[protrusion]{basicmath} % disable protrusion for tt fonts
}{}
\makeatletter
\@ifundefined{KOMAClassName}{% if non-KOMA class
  \IfFileExists{parskip.sty}{%
    \usepackage{parskip}
  }{% else
    \setlength{\parindent}{0pt}
    \setlength{\parskip}{6pt plus 2pt minus 1pt}}
}{% if KOMA class
  \KOMAoptions{parskip=half}}
\makeatother
\usepackage{xcolor}
\IfFileExists{xurl.sty}{\usepackage{xurl}}{} % add URL line breaks if available
\IfFileExists{bookmark.sty}{\usepackage{bookmark}}{\usepackage{hyperref}}
\hypersetup{
  pdftitle={Odev1\_FarukSarikaya},
  pdfauthor={Faruk Sarikaya},
  hidelinks,
  pdfcreator={LaTeX via pandoc}}
\urlstyle{same} % disable monospaced font for URLs
\usepackage[margin=1in]{geometry}
\usepackage{color}
\usepackage{fancyvrb}
\newcommand{\VerbBar}{|}
\newcommand{\VERB}{\Verb[commandchars=\\\{\}]}
\DefineVerbatimEnvironment{Highlighting}{Verbatim}{commandchars=\\\{\}}
% Add ',fontsize=\small' for more characters per line
\usepackage{framed}
\definecolor{shadecolor}{RGB}{248,248,248}
\newenvironment{Shaded}{\begin{snugshade}}{\end{snugshade}}
\newcommand{\AlertTok}[1]{\textcolor[rgb]{0.94,0.16,0.16}{#1}}
\newcommand{\AnnotationTok}[1]{\textcolor[rgb]{0.56,0.35,0.01}{\textbf{\textit{#1}}}}
\newcommand{\AttributeTok}[1]{\textcolor[rgb]{0.77,0.63,0.00}{#1}}
\newcommand{\BaseNTok}[1]{\textcolor[rgb]{0.00,0.00,0.81}{#1}}
\newcommand{\BuiltInTok}[1]{#1}
\newcommand{\CharTok}[1]{\textcolor[rgb]{0.31,0.60,0.02}{#1}}
\newcommand{\CommentTok}[1]{\textcolor[rgb]{0.56,0.35,0.01}{\textit{#1}}}
\newcommand{\CommentVarTok}[1]{\textcolor[rgb]{0.56,0.35,0.01}{\textbf{\textit{#1}}}}
\newcommand{\ConstantTok}[1]{\textcolor[rgb]{0.00,0.00,0.00}{#1}}
\newcommand{\ControlFlowTok}[1]{\textcolor[rgb]{0.13,0.29,0.53}{\textbf{#1}}}
\newcommand{\DataTypeTok}[1]{\textcolor[rgb]{0.13,0.29,0.53}{#1}}
\newcommand{\DecValTok}[1]{\textcolor[rgb]{0.00,0.00,0.81}{#1}}
\newcommand{\DocumentationTok}[1]{\textcolor[rgb]{0.56,0.35,0.01}{\textbf{\textit{#1}}}}
\newcommand{\ErrorTok}[1]{\textcolor[rgb]{0.64,0.00,0.00}{\textbf{#1}}}
\newcommand{\ExtensionTok}[1]{#1}
\newcommand{\FloatTok}[1]{\textcolor[rgb]{0.00,0.00,0.81}{#1}}
\newcommand{\FunctionTok}[1]{\textcolor[rgb]{0.00,0.00,0.00}{#1}}
\newcommand{\ImportTok}[1]{#1}
\newcommand{\InformationTok}[1]{\textcolor[rgb]{0.56,0.35,0.01}{\textbf{\textit{#1}}}}
\newcommand{\KeywordTok}[1]{\textcolor[rgb]{0.13,0.29,0.53}{\textbf{#1}}}
\newcommand{\NormalTok}[1]{#1}
\newcommand{\OperatorTok}[1]{\textcolor[rgb]{0.81,0.36,0.00}{\textbf{#1}}}
\newcommand{\OtherTok}[1]{\textcolor[rgb]{0.56,0.35,0.01}{#1}}
\newcommand{\PreprocessorTok}[1]{\textcolor[rgb]{0.56,0.35,0.01}{\textit{#1}}}
\newcommand{\RegionMarkerTok}[1]{#1}
\newcommand{\SpecialCharTok}[1]{\textcolor[rgb]{0.00,0.00,0.00}{#1}}
\newcommand{\SpecialStringTok}[1]{\textcolor[rgb]{0.31,0.60,0.02}{#1}}
\newcommand{\StringTok}[1]{\textcolor[rgb]{0.31,0.60,0.02}{#1}}
\newcommand{\VariableTok}[1]{\textcolor[rgb]{0.00,0.00,0.00}{#1}}
\newcommand{\VerbatimStringTok}[1]{\textcolor[rgb]{0.31,0.60,0.02}{#1}}
\newcommand{\WarningTok}[1]{\textcolor[rgb]{0.56,0.35,0.01}{\textbf{\textit{#1}}}}
\usepackage{graphicx,grffile}
\makeatletter
\def\maxwidth{\ifdim\Gin@nat@width>\linewidth\linewidth\else\Gin@nat@width\fi}
\def\maxheight{\ifdim\Gin@nat@height>\textheight\textheight\else\Gin@nat@height\fi}
\makeatother
% Scale images if necessary, so that they will not overflow the page
% margins by default, and it is still possible to overwrite the defaults
% using explicit options in \includegraphics[width, height, ...]{}
\setkeys{Gin}{width=\maxwidth,height=\maxheight,keepaspectratio}
% Set default figure placement to htbp
\makeatletter
\def\fps@figure{htbp}
\makeatother
\setlength{\emergencystretch}{3em} % prevent overfull lines
\providecommand{\tightlist}{%
  \setlength{\itemsep}{0pt}\setlength{\parskip}{0pt}}
\setcounter{secnumdepth}{-\maxdimen} % remove section numbering

\title{Odev1\_FarukSarikaya}
\author{Faruk Sarikaya}
\date{12/12/2020}

\begin{document}
\maketitle

\hypertarget{soru-1a}{%
\subsection{Soru 1:a)}\label{soru-1a}}

\textbf{Doğum çeyreği kukla değişkenlerini oluşturunuz. Kaç adet kukla
değişken oluşturmalısınız?}

Bağlımlı Değişken : LWKLYWGE // Log weekly wage

Kukla Değişkenler :EDUC (eğitim yılı), RACE, MARRIED kukla
değişkenlerdir. SMSA Bireyler şehirde yaşıyorsa kullanılacak bir kukla
değişkendir, ayrıca 8 adet lokasyon bazlı kukla değişkenler vardır.
Burada lokasyon bazlı bir veri analizi yapmayacağımız için lokasyona
dayalı kukla değişkenleri kullanmayacağız. AGE ve AGESQ (Yaşın karesi)
denklemde YoB ile başlayan kukla değişkenler, YoB bireyin doğduğu yılı
gösterir. Ayrıca cinsiyet değişkeni sadece erkekleri içerdiği için kukla
değişkenlerin içerisine dahil edilmemiştir.

Toplamda kukla değişken sayısı \textbf{24} adettir. Ancak analizimizde
\textbf{15} adet kukla değişken kullanılabilir.

Değişkenlerin anlaşılabilmesi için DataSet içerisinde verilen kolon
başlıkları değiştirilmiştir.

\(LWKLYWGE = \beta_0 + \beta_1EDUC_i + \beta_2RACE_i + \beta_3MARRIED_i +\)
\(\beta_4YoB20_i + \beta_5YoB21_i +\beta_6YoB22_i + \beta_7YoB23_i + \beta_8YoB24_i + \beta_9YoB25_i + \beta_{10}YoB26_i+\)
\(\beta_{11}YoB27_i + \beta_{12}YoB28_i + \beta_{13}YoB29_i + \beta_{14}AGE_i + \beta_{15}AGEQSQ_i\)

\textbf{1. çeyrekte doğan erkeklerin 1 değeri aldığı kukla değişkeni
araç değişken olarak kullanınız 1. çeyrekte ve diğer çeyreklerde
doğanların ortalama haftalık ücretlerinin orta- lamaları arasındaki
farkı bularak, bu farkın sıfırdan farklı olup olmadığını test ediniz}

\begin{Shaded}
\begin{Highlighting}[]
\NormalTok{lm0 =}\StringTok{ }\KeywordTok{lm}\NormalTok{(PumsTable}\OperatorTok{$}\NormalTok{LWKLYWGE }\OperatorTok{~}\StringTok{ }\NormalTok{PumsTable}\OperatorTok{$}\NormalTok{EDUC }\OperatorTok{+}\StringTok{ }\NormalTok{PumsTable}\OperatorTok{$}\NormalTok{AGE, }\DataTypeTok{data =}\NormalTok{ PumsTable)}
\NormalTok{lm1 =}\StringTok{ }\KeywordTok{lm}\NormalTok{(PumsTable}\OperatorTok{$}\NormalTok{LWKLYWGE }\OperatorTok{~}\StringTok{ }\NormalTok{PumsTable}\OperatorTok{$}\NormalTok{EDUC, }\DataTypeTok{data =}\NormalTok{ PumsTable)}
\NormalTok{lm2 =}\StringTok{ }\KeywordTok{lm}\NormalTok{(PumsTable}\OperatorTok{$}\NormalTok{LWKLYWGE }\OperatorTok{~}\StringTok{ }\NormalTok{PumsTable}\OperatorTok{$}\NormalTok{AGE, }\DataTypeTok{data =}\NormalTok{ PumsTable)}
\NormalTok{lm3 =}\StringTok{ }\KeywordTok{lm}\NormalTok{(PumsTable}\OperatorTok{$}\NormalTok{EDUC }\OperatorTok{~}\StringTok{ }\NormalTok{PumsTable}\OperatorTok{$}\NormalTok{AGE, }\DataTypeTok{data =}\NormalTok{ PumsTable)}
\KeywordTok{stargazer}\NormalTok{(lm0,lm1,lm2,lm3,}\DataTypeTok{type =} \StringTok{"html"}\NormalTok{, }\DataTypeTok{style =} \StringTok{"qje"}\NormalTok{)}
\end{Highlighting}
\end{Shaded}

\begin{verbatim}
## 
## <table style="text-align:center"><tr><td colspan="5" style="border-bottom: 1px solid black"></td></tr><tr><td style="text-align:left"></td><td colspan="3">LWKLYWGE</td><td>EDUC</td></tr>
## <tr><td style="text-align:left"></td><td>(1)</td><td>(2)</td><td>(3)</td><td>(4)</td></tr>
## <tr><td colspan="5" style="border-bottom: 1px solid black"></td></tr><tr><td style="text-align:left">EDUC</td><td>0.168<sup>***</sup></td><td>0.296<sup>***</sup></td><td></td><td></td></tr>
## <tr><td style="text-align:left"></td><td>(0.002)</td><td>(0.001)</td><td></td><td></td></tr>
## <tr><td style="text-align:left"></td><td></td><td></td><td></td><td></td></tr>
## <tr><td style="text-align:left">AGE</td><td>0.00000<sup>***</sup></td><td></td><td>0.00000<sup>***</sup></td><td>0.00000<sup>***</sup></td></tr>
## <tr><td style="text-align:left"></td><td>(0.000)</td><td></td><td>(0.000)</td><td>(0.000)</td></tr>
## <tr><td style="text-align:left"></td><td></td><td></td><td></td><td></td></tr>
## <tr><td style="text-align:left">Constant</td><td>8.966<sup>***</sup></td><td>8.985<sup>***</sup></td><td>9.203<sup>***</sup></td><td>1.410<sup>***</sup></td></tr>
## <tr><td style="text-align:left"></td><td>(0.003)</td><td>(0.003)</td><td>(0.001)</td><td>(0.001)</td></tr>
## <tr><td style="text-align:left"></td><td></td><td></td><td></td><td></td></tr>
## <tr><td style="text-align:left"><em>N</em></td><td>1,063,634</td><td>1,063,634</td><td>1,063,634</td><td>1,063,634</td></tr>
## <tr><td style="text-align:left">R<sup>2</sup></td><td>0.071</td><td>0.044</td><td>0.060</td><td>0.233</td></tr>
## <tr><td style="text-align:left">Adjusted R<sup>2</sup></td><td>0.071</td><td>0.044</td><td>0.060</td><td>0.233</td></tr>
## <tr><td style="text-align:left">Residual Std. Error</td><td>0.768 (df = 1063631)</td><td>0.780 (df = 1063632)</td><td>0.773 (df = 1063632)</td><td>0.495 (df = 1063632)</td></tr>
## <tr><td style="text-align:left">F Statistic</td><td>40,635.220<sup>***</sup> (df = 2; 1063631)</td><td>48,814.620<sup>***</sup> (df = 1; 1063632)</td><td>68,067.500<sup>***</sup> (df = 1; 1063632)</td><td>322,364.500<sup>***</sup> (df = 1; 1063632)</td></tr>
## <tr><td colspan="5" style="border-bottom: 1px solid black"></td></tr><tr><td style="text-align:left"><em>Notes:</em></td><td colspan="4" style="text-align:right"><sup>***</sup>Significant at the 1 percent level.</td></tr>
## <tr><td style="text-align:left"></td><td colspan="4" style="text-align:right"><sup>**</sup>Significant at the 5 percent level.</td></tr>
## <tr><td style="text-align:left"></td><td colspan="4" style="text-align:right"><sup>*</sup>Significant at the 10 percent level.</td></tr>
## </table>
\end{verbatim}

\end{document}
